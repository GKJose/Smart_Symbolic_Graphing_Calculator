\documentclass[a4paper,11pt]{book}
\usepackage{amsmath}
\usepackage{amsfonts}
\usepackage{amssymb}
\usepackage{stmaryrd}
\usepackage{makeidx}
\usepackage{times}
\newcommand{\R}{{\mathbb{R}}}
\newcommand{\C}{{\mathbb{C}}}
\newcommand{\Z}{{\mathbb{Z}}}
\newcommand{\N}{{\mathbb{N}}}
\newcommand{\asinh}{\,\,\mbox{asinh\,}}
\newcommand{\atanh}{\,\,\mbox{atanh\,}}
\begin{document}
\chapter{Graphs}\label{sec:plot}
Most graph instructions take expressions as arguments. A few
exceptions (mostly maple-compatibility instructions) also accept
functions. 
Some optional arguments, like {\tt color, thickness}, can  be used as optional
attributes in all graphic instructions. They are described below.

\section{Graph and geometric objects attributes}
There are two kinds of attributes: global attributes of a graphic
scene and individual attributes. 

\subsection{Individual attributes}\index{color}\index{display}
\index{red@{\it red}|textbf}\index{blue@{\it blue}|textbf}\index{yellow@{\it yellow}|textbf}\index{magenta@{\it magenta}|textbf}\index{green@{\it green}|textbf}\index{cyan@{\it cyan}|textbf}\index{white@{\it white}|textbf}\index{black@{\it black}|textbf}\index{filled@{\it filled}}
Graphic attributes are optional arguments of the
form {\tt display=value}, they must be given
as last argument of a graphic instruction. Attributes
are ordered in several categories: color, point shape, point width,
line style, line thickness, legend value, position and presence. 
In addition, surfaces may be filled or not, 3-d surfaces
may be filled with a texture, 3-d objects may also have properties
with respect to the light. 
Attributes of different categories
may be added, e.g. \\
{\tt plotfunc($x^2+y^2$,[x,y],display=red+line\_width\_3+filled}
\begin{itemize}
\item Colors {\tt display=} or {\tt color=}
\begin{itemize}
\item {\tt black}, {\tt white}, {\tt red}, {\tt blue}, {\tt green}, 
{\tt magenta}, {\tt cyan}, {\tt yellow},
\item a numeric value between 0 and 255,
\item a numeric value between 256 and 256+7*16+14 for a color of the
rainbow,
\item any other numeric value smaller than 65535, the rendering
is not garanteed to be portable.
\end{itemize}
\item Point shapes {\tt display=} one of the following value
{\tt rhombus\_point plus\_point  square\_point cross\_point 
triangle\_point star\_point point\_point invisible\_point}
\item Point width: {\tt display=} one of the following value
{\tt point\_width\_n} where {\tt n} is an
integer between 1 and 7
\item Line thickness: {\tt thickness=n}
or {\tt display=line\_width\_n} where {\tt n} is an
integer between 1 and 7 or 
\item Line shape: {\tt display=} one of the following value
{\tt dash\_line solid\_line dashdot\_line dashdotdot\_line
  cap\_flat\_line cap\_square\_line cap\_round\_line }
\item Legend, value: {\tt legend="legendname"};
 position: {\tt display=} one of
{\tt quandrant1 quadrant2 quadrant3 quadrant4}
corresponding to the position of the legend of the object 
(using the trigonometric plan conventions).
The legend is not displayed if the attribute 
{\tt display=hidden\_name} is added
\item {\tt display=filled} specifies that surfaces will be filled
\item {\tt gl\_texture="picture\_filename"} is used to fill 
a surface with a texture.  
Cf. the interface manual for a more complete
description and for {\tt gl\_material=} options.
\end{itemize}
{\bf Examples}
Input
\begin{center}{\tt polygone(-1,-i,1,2*i,legende="P")}\end{center}
Input
\begin{center}{\tt point(1+i,legende="bonjour")}\end{center}
Input
\begin{center}{\tt A:=point(1+i);B:=point(-1);affichage(D:=droite(A,B),hidden\_name)}\end{center}
Input
\begin{center}{\tt couleur(segment(0,1+i),rouge)}\end{center}
Input
\begin{center}{\tt segment(0,1+i,couleur=rouge)}\end{center}

\subsection{Global attributes}
These attributes are shared by all objets of the same scene
\begin{itemize}
\item {\tt title="titlename"} defines the title
\item {\tt labels=["xname","yname","zname"]}: names of the $x,y,z$
axis
\item {\tt gl\_x\_axis\_name="xname"}, {\tt gl\_y\_axis\_name="yname"},
{\tt gl\_z\_axis\_name=""}: individual definition
of the names of the $x,y,z$ axis
\item {\tt legend=["xunit","yunit","zunit"]}: units for the
$x,y,z$ axis
\item {\tt gl\_x\_axis\_unit="xunit"}, {\tt  gl\_y\_axis\_unit="yunit"},
{\tt gl\_z\_axis\_unit=""}: individual definition
of the units of the $x,y,z$ axis
\item {\tt axes=true or false} show or hide axis
\item {\tt gl\_texture="filename"}: background image
\item {\tt gl\_x=xmin..xmax}, {\tt gl\_y=ymin..ymax},
{\tt gl\_z=zmin..zmax}: set the graphic configuration 
(do not use for interactive scenes)
\item {\tt gl\_xtick=}, {\tt gl\_ytick=}, {\tt gl\_ztick=}:
set the tick mark for the axis 
\item {\tt gl\_shownames=true or false}: show or hide objects names
\item {\tt gl\_rotation=[x,y,z]}: defines the rotation axis
for the animation rotation of 3-d scenes�
\item {\tt gl\_quaternion=[x,y,z,t]}: defines the quaternion
for the visualization in 3-d scenes (do not use for interactive
scenes)
\item a few other OpenGL light configuration options are
available but not described here.
\end{itemize}
{\bf Examples}
Input
\begin{center}{\tt legende=["mn","kg"]}\end{center}
Input
\begin{center}{\tt titre="medianes";triangle(-1-i,1,1+i);mediane(-1-i,1,1+i);mediane(1,-1-i,1+i);mediane(1+i,1,-1-i)}\end{center}
Input
\begin{center}{\tt labels=["u","v"];plotfunc(u+1,u)}\end{center}
\section{Graph of a function : {\tt plotfunc funcplot DrawFunc Graph}}\index{plotfunc|textbf}\index{funcplot|textbf}\index{DrawFunc|textbf}\index{Graph|textbf}\index{xstep@{\sl xstep}}\index{ystep@{\sl ystep}}\index{zstep@{\sl zstep}}\index{nstep@{\sl nstep}}

\subsection{2-d graph}\label{sec:plotfunc}
\noindent{\tt plotfunc(f(x),x)} draws the graph of $y=f(x)$ for $x$ in
the default interval, 
{\tt plotfunc(f(x),x=a..b)} draws the graph of $y=f(x)$ for $a\leq x\leq b$.
{\tt plotfunc} accepts an optional \verb|xstep=...| argument to 
specify the discretisation step in $x$.\\
Input :
\begin{center}{\tt  plotfunc(x\verb|^|2-2)}\end{center}
or
\begin{center}{\tt  plotfunc(a\verb|^|2-2,a=-1..2)}\end{center}
Output :
\begin{center}{\tt the graph of y=x\verb|^|2-2}\end{center}
Input :
\begin{center}{\tt  plotfunc(x\verb|^|2-2,x,xstep=1)}\end{center}
Output :
\begin{center}{\tt a polygonal line which is a bad representation of y=x\verb|^|2-2 }\end{center}
It is also possible to specify the number of points used for the 
representation of the function with \verb|nstep=| instead of \verb|xstep=|.
For example, input~:
\begin{center}{\tt  plotfunc(x\verb|^|2-2,x=-2..3,nstep=30)}\end{center}

\subsection{3-d graph}\label{sec:plotfunc3}
\noindent{\tt plotfunc} takes two main arguments : an expression of two 
variables or a list of several expressions of two variables and the list of 
these two variables, where each variable may be replaced by
an equality variable=interval to specify the range for this variable
(if not specified, default values are taken from the graph configuration).
{\tt plotfunc} accepts two optional arguments to specify 
the discretisation step in $x$ and in $y$ by
{\tt xstep=...} and {\tt ystep=...}.
Alternatively one can specify the number of points used for the 
representation of the function with \verb|nstep=| (instead of \verb|xstep| and 
{\tt ystep}).\\
{\tt plotfunc} draws the surface(s) defined by $z=$ the first argument.\\
Input :
\begin{center}{\tt plotfunc( x\verb|^|2+y\verb|^|2,[x,y])}\end{center}
Output :
\begin{center}{\tt A 3D graph of z=x\verb|^|2+y\verb|^|2}\end{center}
Input :
\begin{center}{\tt plotfunc(x*y,[x,y]) }\end{center}
Output :
\begin{center}{\tt The surface z=x*y, default ranges}\end{center}
Input :
\begin{center}{\tt plotfunc([x*y-10,x*y,x*y+10],[x,y]) }\end{center}
Output :
\begin{center}{\tt The surfaces z=x*y-10, z=x*y and z=x*y+10}\end{center}
Input :
\begin{center}{\tt plotfunc(x*sin(y),[x=0..2,y=-pi..pi]) }\end{center}
Output :
\begin{center}{\tt The surface $z=x*y$ for the specified ranges}\end{center}
Now an example where we specify the $x$ and $y$ discretisation step 
with \verb|xstep| and \verb|ystep|. Input :
\begin{center}
{\tt plotfunc(x*sin(y),[x=0..2,y=-pi..pi],xstep=1,ystep=0.5) }\end{center}
Output :
\begin{center}{\tt A portion of surface $z=x*y$}\end{center}
Alternatively we may specify 
the number of points used for the representation of the
function with \verb|nstep| instead of \verb|xstep| and \verb|ystep|, input~:
\begin{center}{\tt plotfunc(x*sin(y),[x=0..2,y=-pi..pi],nstep=300)}\end{center}
Output :
\begin{center}{\tt A portion of surface $z=x*y$}\end{center}
{\bf Remarks}
\begin{itemize}
\item
Like any 3-d scene, the viewpoint may be modified by rotation 
around the {\tt x} axis, the {\tt y} axis or the
{\tt z} axis, either by dragging the mouse inside the graphic 
window (push the mouse outside the parallelepiped used for 
the representation), or with the shortcuts
{\tt x}, {\tt X}, {\tt y}, {\tt Y}, {\tt z} and {\tt Z}.
\item
If you want to print a graph or get a \LaTeX\ translation, use the graph
menu\\
{\tt Menu$\blacktriangleright$print$\blacktriangleright$Print(with
  Latex)}
\end{itemize}

\subsection{3-d graph with rainbow colors}\label{sec:plotfunc3d}
\noindent{\tt plotfunc} represents a pure imaginary expression {\tt i*E}
of two variables with a rainbow color depending 
on the value of {\tt z=E}. This gives an easy way to 
find points having the same third coordinate.\\
The first arguments of {\tt plotfunc} must be {\tt i*E} instead of {\tt E},
the remaining arguments are the same 
as for a real 3-d graph (cf \ref{sec:plotfunc3})
Input :
\begin{center}{\tt plotfunc(i*x*sin(y),[x=0..2,y=-pi..pi]) }\end{center}
Output :
\begin{center}{\tt A piece of the surface $z=x*\sin(y)$ with rainbow colors}\end{center}
{\bf Remark}\\
 If you want the impression or the  traduction Latex, you have to use :\\
{\tt Menu$\blacktriangleright$print$\blacktriangleright$Print(with Latex)}. 

\subsection{4-d graph.}\label{sec:plotfunc4}
\noindent{\tt plotfunc} represents a complex expression {\tt E} 
(such that {\tt re(E)} is not identically 0 on the discretisation mesh)
by the surface {\tt z=abs(E)} where {\tt arg(E)} defines the color 
from the rainbow. This gives an easy way to 
see the points having the same argument.
Note that if {\tt re(E)==0} on the discretisation mesh, 
it is the surface {\tt z=E/i} that is represented with rainbow colors 
(cf \ref{sec:plotfunc3d}).\\
The first argument of {\tt plotfunc} is {\tt E}, 
the remaining arguments are the same 
as for a real 3-d graph (cf \ref{sec:plotfunc3}).
Input :
\begin{center}{\tt plotfunc((x+i*y)\verb|^|2,[x,y])}\end{center}
Output :
\begin{center}{\tt A graph 3D of z=abs((x+i*y)\verb|^|2 with the same color for
points having the same argument}\end{center}
Input :
\begin{center}{\tt plotfunc((x+i*y)\verb|^|2x,[x,y], display=filled)}\end{center}
Output :
\begin{center}{\tt The same surface but filled}\end{center}
We may specify the range of variation of $x$ and $y$ and the number of 
discretisation points, input :
\begin{center}{\tt plotfunc((x+i*y)\verb|^|2,[x=-1..1,y=-2..2], nstep=900,display=filled)}\end{center}
Output :
\begin{center}{\tt The specified part of the surface with $x$ between -1 and 1, $y$ between -2 and 2 and with 900 points}\end{center}
 
\section{2d graph for Maple compatibility : {\tt plot}}
\index{plot} \label{sec:plot2d}
\noindent{\tt plot(f(x),x)} draws the graph of $y=f(x)$. 
The second argument may specify the range of values {\tt
  x=xmin..xmax}. One can also plot a function instead of an
expression using the syntax {\tt plot(f,xmin..xmax)}.
{\tt plot} accepts an optional argument to specify 
the step used in $x$ for the discretisation with  
\verb|xstep=| or the number of points of the discretization
with \verb|nstep=|.\\
Input :
\begin{center}{\tt  plot(x\verb|^|2-2,x)}\end{center}
Output :
\begin{center}{\tt the graph of y=x\verb|^|2-2}\end{center}
Input :
\begin{center}{\tt  plot(x\verb|^|2-2,xstep=1)}\end{center}
or 
\begin{center}{\tt  plot(x\verb|^|2-2,x,xstep=1)}\end{center}
Output :
\begin{center}{\tt a polygonal line which is a bad representation of
    y=x\verb|^|2-2 }\end{center}
Input!
\begin{center}{\tt  plot(x\verb|^|2-2,x=-2..3,nstep=30)}\end{center}


\section{3d surfaces for Maple compatibility {\tt plot3d}}\index{plot3d}
\noindent{\tt plot3d} takes three arguments : a function of two variables or 
an expression of two variables  or a list of three functions of two variables 
or a list of three expressions of two variables and the names of these two 
variables with an optional range (for expressions) or the ranges 
(for functions).\\
{\tt plot3d(f(x,y),x,y)} (resp {\tt plot3d([f(u,v),g(u,v),h(u,v)],u,v)}) draws 
the surface $z=f(x,y)$ (resp $x=f(u,v),y=g(u,v),z=h(u,v)$).
The syntax {\tt plot3d(f(x,y),x=x0..x1,y=y0..y1)} or 
{\tt plot3d(f,x0..x1,y0..y1)} specifies which part of surface 
will be computed (otherwise default values are taken from the graph
configuration).\\ 
Input :
\begin{center}{\tt plot3d(x*y,x,y)}\end{center}
Output :
\begin{center}{\tt The surface $z=x*y$}\end{center}
Input :
\begin{center}{\tt plot3d([v*cos(u),v*sin(u),v],u,v) }\end{center}
Output :
\begin{center}{\tt The cone $x=v*\cos(u),y=v*\sin(u),z=v$}\end{center}
Input :
\begin{center}{\tt plot3d([v*cos(u),v*sin(u),v],u=0..pi,v=0..3)}\end{center}
Output :
\begin{center}{\tt A portion of the cone $x=v*\cos(u),y=v*\sin(u),z=v$}\end{center}

\section{Graph of a line and tangent to a graph}
\subsection{Draw a line : {\tt line}}\index{line}\label{sec:doite}
{\bf See also :} \ref{sec:droite2} and \ref{sec:droite3} for line usage in 
geometry and see \ref{sec:axe2} and \ref{sec:axe3} for axis.\\
\noindent {\tt line} takes as argument cartesian(s) equation(s) :
\begin{itemize}
\item in 2D: one line equation,
\item in 3D: two plane equations.
\end{itemize}
{\tt line} defines and draws the corresponding line.\\
Input :
\begin{center}{\tt line(2*y+x-1=0)}\end{center}
Output :
\begin{center}{\tt the line 2*y+x-1=0}\end{center}
Input :
\begin{center}{\tt line(y=1)}\end{center}
Output :
\begin{center}{\tt the horizontal line y=1}\end{center}
Input :
\begin{center}{\tt line(x=1)}\end{center}
Output :
\begin{center}{\tt the vertical line x=1}\end{center}
Input :
\begin{center}{\tt line(x+2*y+z-1=0,z=2)}\end{center}
Output :
\begin{center}{\tt the line x+2*y+1=0 in the plane z=2}\end{center}
Input :
\begin{center}{\tt line(y=1,x=1)}\end{center}
Output :
\begin{center}{\tt the vertical line crossing through (1,1,0)}\end{center}
{\bf Remark}\\
{\tt line} defines an oriented line :
\begin{itemize}
\item when the 2D line is given by an equation, it is rewritten
as "left\_member-right\_member={\tt ax+by+c=0}", this determinates
its normal vector {\tt [a,b]} and the orientation is given by the vector 
{\tt [b,-a]}) (or its orientation is defined by the 3D cross product of its
normal vectors (with third coordinate 0) and the vector [0,0,1]).\\
For example {\tt line(y=2*x)} defines the line {\tt -2x+y=0} with as direction 
the vector {\tt [1,2]} (or {\tt cross([-2,1,0],[0,0,1])}={\tt [1,2,0]}).
\item when the 3D line is given by two plane equations, it's 
direction is defined by the cross product of the normals to the planes 
(where the plane equation is rewritten as
"left\_member-right\_member={\tt ax+by+cz+d=0}", so that
the normal is {\tt [a,b,c]}).\\
For example the {\tt line(x=y,y=z)} is the line {\tt x-y=0,y-z=0} and its
direction is :\\
{\tt cross([1,-1,0],[0,1,-1])}={\tt [1,1,1]}.
\end{itemize}

\subsection{Draw an 2D horizontal line : {\tt LineHorz}}\index{LineHorz}
\noindent {\tt LineHorz} takes as argument an expression $a$.\\
 {\tt LineHorz} draws the horizontal line $y=a$.\\
Input :
\begin{center}{\tt LineHorz(1)}\end{center}
Output :
\begin{center}{\tt the line y=1}\end{center}

\subsection{Draw a 2D vertical line : {\tt LineVert}}\index{LineVert}
\noindent {\tt LineVert} takes as argument an expression $a$.\\
 {\tt LineVert} draws the vertical line $x=a$.\\
Input :
\begin{center}{\tt LineVert(1)}\end{center}
Output :
\begin{center}{\tt the line x=1}\end{center}

\subsection{Tangent to a 2D graph : {\tt LineTan}}\index{LineTan}
\noindent {\tt LineTan} takes two arguments : an expression $E_x$ of the
variable $x$ and a value $x0$ of $x$.\\
 {\tt LineTan} draws the tangent at $x=x0$ to the graph of $y=E_x$.\\
Input :
\begin{center}{\tt LineTan(ln(x),1)}\end{center}
Output :
\begin{center}{\tt the line y=x-1}\end{center}
Input :
\begin{center}{\tt equation(LineTan(ln(x),1))}\end{center}
Output :
\begin{center}{\tt y=(x-1)}\end{center}

\subsection{Tangent to a 2D graph : {\tt tangent}}\index{tangent|textbf}\label{sec:tangente}
{\bf See also :} \ref{sec:tangent} for plane geometry and 
\ref{sec:tangent3} for 3D geometry.\\
\noindent {\tt tangent} takes two arguments : an geometric object and a point 
{\tt A}.\\
{\tt tangent} draws tangent(s) to this geometric object crossing through 
{\tt A}. If the geometric object is the graph {\tt G} of a 2D function, 
the second argument is either, a real number {\tt x0}, or a 
point {\tt A} on {\tt G}. In that case {\tt tangent} draws a tangent to this
graph {\tt G} crossing through the point {\tt A} or through the 
point of abscissa {\tt x0}.\\
For example, define the function {\tt g}
\begin{center} \verb|g(x):=x^2|\end{center}
then the graph {\tt G=\{(x,y)$\in \R^2$, y=g(x)\}}
of $g$ and a point $A$ on the graph $G$:
\begin{center}
{\tt G:=plotfunc(g(x),x);}\\
{\tt A:=point(1.2,g(1.2));}
\end{center}
If we want to draw the tangent at the point {\tt A} to the graph {\tt
  G}, we will input:
\begin{center}
{\tt T:=tangent(G, A)}
\end{center}
or :
\begin{center}
{\tt T:=tangent(G, 1.2)}
\end{center}
For the equation of the tangent line, input :
\begin{center}{\tt equation(T)}\end{center}

\subsection{Intersection of a 2D graph with the axis}\index{solve}\index{resoudre}
\begin{itemize}
\item The ordinate of the intersection of the graph of $f$ with the 
$y$-axis is returned by :
\begin{center}{\tt f(0)}\end{center}
indeed the point of coordinates $(0,f(0))$ is the intersection point of the 
graph of $f$ with the $y$-axis,
\item Finding the intersection of the graph of $f$ with the $x$-axis 
requires to solve the equation $f(x)=0$. \\
If the equation is polynomial-like, {\tt solve} will find
the exact values of the abscissa of these points. Input:
\begin{center}{\tt solve(f(x),x)}\end{center}
Otherwise, we can find numeric approximations of these 
abscissa. First look at the graph for an initial guess or a
range with an intersection and refine with {\tt fsolve}.
\end{itemize}

\section{Graph of inequations with 2 variables : {\tt plotinequation inequationplot}}\index{plotinequation|textbf}\index{inequationplot|textbf}
\noindent{\tt plotinequation([f1(x,y)<a1,...fk(x,y)<ak],[x=x1..x2,y=y1..y2])} 
draws the points of the plane whose coordinates
satisfy the inequations of 2 variables :
\[ \left\{ \begin{array}{ccc}
f1(x,y) &<&a1 \\
& ... & \\
fk(x,y)&<&ak 
\end{array}\right., \quad
x1\leq x \leq x2, y1 \leq y \leq y2 \]
Input :
\begin{center}{\tt plotinequation(x\verb|^|2-y\verb|^|2<3, [x=-2..2,y=-2..2],xstep=0.1,ystep=0.1)}\end{center}
Output :
\begin{center}{\tt the filled portion enclosing the origin and limited by the hyperbola x\verb|^|2-y\verb|^|2=3}\end{center}
Input :
\begin{center}{\tt plotinequation([x+y>3,x\verb|^|2<y], [x-2..2,y=-1..10],xstep=0.2,ystep=0.2)}\end{center}
Output :
\begin{center}{\tt the filled portion of the plane defined by -2<x<2,y<10,x+y>3,y>x\verb|^|2}\end{center}
Note that if the ranges for $x$ and $y$ are not specified, 
{\tt Xcas} takes the default values of 
{\tt X-,X+,Y-,Y+} defined in the general graphic configuration
({\tt Cfg$\blacktriangleright$Graphic configuration}).

\section{Graph of the area below a curve : {\tt plotarea areaplot}}\index{plotarea|textbf}\index{areaplot|textbf}\index{rectangle\_droit@{\sl rectangle\_droit}|textbf}\index{rectangle\_gauche@{\sl rectangle\_gauche}|textbf}\index{trapeze@{\sl trapeze}|textbf}\index{point\_milieu@{\sl point\_milieu}|textbf}
\begin{itemize}
\item With two arguments, {\tt plotarea} shades the area below a curve.\\ 
{\tt plotarea(f(x),x=a..b)} draws the area below the curve $y=f(x)$ for 
$a<x<b$, i.e. the portion of the plane defined by the inequations $a<x<b$ and
$0<y<f(x)$ or $0>y>f(x)$ according to the sign of $f(x)$ .\\
Input :
\begin{center}{\tt plotarea(sin(x),x=0..2*pi)}\end{center}
Output :
\begin{center}{\tt the portion of plane locates in the two archs of sin(x)}\end{center}
\item With four arguments, {\tt plotarea}  represents a numeric approximation
of the area below a curve, according to a quadrature method from the
following list:\\
{\tt trapezoid,rectangle\_left,rectangle\_right,middle\_point}.\\
For example {\tt plotarea(f(x),x=a..b,n,trapezoid)} 
draws the area of $n$ trapezes : the 
third argument is an integer $n$, and the fourth argument is the name of the 
numeric method of integration when $[a,b]$ is cut into $n$ equal parts.\\
Input :
\begin{center}{\tt plotarea((x\verb|^|2,x=0..1,5,trapezoid)}\end{center}
If you want to display the graph of the curve in contrast
(e.g. in bold red), input :
\begin{center}{\tt plotarea(x\verb|^|2,x=0..1,5,trapezoid); 
plot(x\verb|^|2,x=0..1,display=red+line\_width\_3)}\end{center}
Output :
\begin{center}{\tt the 5 trapezes used in the trapezoid method to approach the integral}\end{center}
Input :
\begin{center}{\tt plotarea((x\verb|^|2,x=0..1,5,middle\_point)}\end{center}
Or with the graph of the curve in bold red, input :
\begin{center}{\tt plotarea(x\verb|^|2,x=0..1,5,middle\_point); plot(x\verb|^|2,x=0..1,display=red+line\_width\_3)}\end{center}
Output :
\begin{center}{\tt the 5 rectangles used in the middle\_point method
    to approach the integral}\end{center}
\end{itemize}

\section{Contour lines: {\tt plotcontour contourplot \\DrwCtour}}\index{plotcontour|textbf}\index{contourplot|textbf}\index{DrwCtour|textbf}\label{sec:plotcontour}
\noindent{\tt plotcontour(f(x,y),[x,y])} (or {\tt DrwCtour(f(x,y),[x,y])} or \\
 {\tt contourplot(f(x,y),[x,y])})
draws the contour lines of the surface defined by $z=f(x,y)$ for $z=-10$, 
$z=-8$, .., $z=0$, $z=2$, .., $z=10$. You may specify the desired contour
lines by a list of values of $z$ given as third argument.\\
Input :
\begin{center}{\tt  plotcontour(x\verb|^|2+y\verb|^|2,[x=-3..3,y=-3..3],[1,2,3], display=[green,red,black]+[filled\$3])}\end{center}
Output :
\begin{center}{\tt  the graph of the three ellipses x\verb|^|2-y\verb|^|2=n for n=1,2,3; the zones between these ellipses are filled with the color green,red or black}\end{center}
Input :
\begin{center}{\tt  plotcontour(x\verb|^|2-y\verb|^|2,[x,y])}\end{center}
Output :
\begin{center}{\tt  the graph of 11 hyperbolas x\verb|^|2-y\verb|^|2=n for n=-10,-8,..10}\end{center}

If you want to draw the surface in 3-d representation, 
input {\tt plotfunc(f(x,y),[x,y])}, see \ref{sec:plotfunc3}):
\begin{center}{\tt plotfunc( x\verb|^|2-y\verb|^|2,[x,y])}\end{center}
Output :
\begin{center}{\tt A 3D representation of z=x\verb|^|2+y\verb|^|2}\end{center}

\section{2-d graph of a 2-d function with colors : 
{\tt plotdensity densityplot}}
\index{plotdensity|textbf}\index{densityplot|textbf}
\noindent{\tt plotdensity(f(x,y),[x,y])}  or  {\tt densityplot(f(x,y),[x,y])}
draws the graph of $z=f(x,y)$ in the plane where the values of
$z$ are represented by the rainbow colors. The optional argument
{\tt z=zmin..zmax} specifies the range of $z$ corresponding to the
full rainbow, if it is not specified, it is deduced from the minimum
and maximum value of $f$ on the discretisation. The discretisation
may be specified by optional {\tt xstep=...} and {\tt ystep=...}
or {\tt nstep=...} arguments.\\
Input :
\begin{center}{\tt plotdensity(x\verb|^|2-y\verb|^|2,[x=-2..2,y=-2..2], xstep=0.1,ystep=0.1)}\end{center}
Output :
\begin{center}{\tt A 2D graph where each hyperbola defined by
    x\verb|^|2-y\verb|^|2=z has a color from the rainbow}\end{center}
{\bf Remark} : A rectangle representing the scale of colors is 
displayed below the graph.

\section{Implicit graph: {\tt plotimplicit implicitplot}}\index{plotimplicit}\index{implicitplot}\index{unfactored}
\noindent{\tt plotimplicit} or {\tt implicitplot} draws curves or 
surfaces defined by an implicit expression or equation. 
If the option {\tt unfactored} is given as last argument, the
original expression is taken unmodified. Otherwise,
the expression is normalized, then replaced by the
factorization of the numerator of it's normalization.

Each factor of the expression corresponds to a component
of the implicit curve or surface. For each factor,
Xcas tests if it is of total
degree less or equal to 2, in that case {\tt conic} or
{\tt quadric} is called. Otherwise the numeric implicit solver
is called. 

Optional step and ranges arguments may be passed to the numeric
implicit solver, note that they are dismissed for each component
that is a conic or a quadric.

\subsection{2D implicit curve}\label{sec:implicitplot}
\begin{itemize}
\item {\tt plotimplicit(f(x,y),x,y)} draws the graphic representation of the
curve defined by the implicit equation $f(x,y)=0$ when $x$ (resp $y$) 
is in {\tt WX-, WX+} (resp in {\tt WY-, WY+}) defined by {\tt cfg}, 

\item {\tt plotimplicit(f(x,y),x=0..1,y=-1..1)} draws the graphic 
representation of the curve defined by the implicit equation $f(x,y)=0$ 
when $0\leq x \leq 1$ and $-1\leq y \leq 1$
\end{itemize} 
It is possible to add two arguments to specify the discretisation
steps for $x$ 
and $y$ with {\tt xstep=...} and {\tt ystep=...}.\\
Input :
\begin{center}{\tt plotimplicit(x\verb|^|2+y\verb|^|2-1,x,y)}\end{center}
Or :
\begin{center}{\tt plotimplicit(x\verb|^|2+y\verb|^|2-1,x,y,unfactored)}\end{center}
Output :
\begin{center}{\tt The unit circle}\end{center}
Input :
\begin{center}{\tt plotimplicit(x\verb|^|2+y\verb|^|2-1,x,y,xstep=0.2,ystep=0.3)}\end{center}
Or :
\begin{center}{\tt plotimplicit(x\verb|^|2+y\verb|^|2-1,[x,y],xstep=0.2,ystep=0.3)}\end{center}
Or :
\begin{center}{\tt plotimplicit(x\verb|^|2+y\verb|^|2-1,[x,y], xstep=0.2,ystep=0.3,unfactored)}\end{center}
Output :
\begin{center}{\tt The unit circle}\end{center}
Input :
\begin{center}{\tt plotimplicit(x\verb|^|2+y\verb|^|2-1,x=-2..2,y=-2..2, xstep=0.2,ystep=0.3)}\end{center}
Output :
\begin{center}{\tt The unit circle}\end{center}

\subsection{3D implicit surface}\label{sec:implicitplot3}
\begin{itemize}
\item {\tt plotimplicit(f(x,y,z),x,y,z)} draws the graphic 
representation of the surface defined by the implicit equation $f(x,y,z)=0$, 
\item {\tt plotimplicit(f(x,y,z),x=0..1,y=-1..1,z=-1..1)} draws the surface 
defined by the implicit equation $f(x,y,z)=0$, 
where $0\leq x \leq 1$, $-1\leq y \leq 1$ and $-1\leq z \leq 1$.
\end{itemize}
It is possible to add three arguments to specify the discretisation
steps used for $x$, $y$ and $z$ with {\tt xstep=...}, {\tt ystep=...} and 
{\tt zstep=...}.\\
Input :
\begin{center}{\tt plotimplicit(x\verb|^|2+y\verb|^|2+z\verb|^|2-1,x,y,z, xstep=0.2,ystep=0.1,zstep=0.3)}\end{center}
Input :
\begin{center}{\tt plotimplicit(x\verb|^|2+y\verb|^|2+z\verb|^|2-1,x,y,z, xstep=0.2,ystep=0.1,zstep=0.3,unfactored)}\end{center}
Output :
\begin{center}{\tt The unit sphere}\end{center}
Input :
\begin{center}{\tt plotimplicit(x\verb|^|2+y\verb|^|2+z\verb|^|2-1,x=-1..1,y=-1..1,z=-1..1)}\end{center}
Output :
\begin{center}{\tt The unit sphere}\end{center}

\section{Parametric curves and surfaces : {\tt plotparam paramplot DrawParm}}\index{plotparam|textbf}\index{paramplot|textbf}\index{DrawParm|textbf}
\subsection{2D parametric curve }
\noindent {\tt plotparam([f(t),g(t)],t)}
or {\tt plotparam(f(t)+i*g(t),t)} (resp 
{\tt plotparam(f(t)+i*g(t),t=t1..t2)})
draws the parametric representation of the curve 
defined by  $x=f(t),y=g(t)$ 
with the default range of values of $t$ (resp for $t1 \leq t\leq t2$).\\
The default range of values is taken as specified 
in the graphic configuration ({\tt t-} and {\tt t+}, 
cf. \ref{sec:configgeo}).
{\tt plotparam} accepts an optional argument to specify the discretisation
step for $t$ with {\tt tstep=}.\\ 
Input :
\begin{center}{\tt plotparam(cos(x)+i*sin(x),x) }\end{center}
or :
\begin{center}{\tt plotparam([cos(x),sin(x)],x) }\end{center}
Output :
\begin{center}{\tt The unit circle}\end{center}
If in the graphic configuration {\tt t} goes from -4 to 1, input :
\begin{center}{\tt plotparam(sin(t)+i*cos(t))}\end{center}
or :
\begin{center}{\tt plotparam(sin(t)+i*cos(t),t=-4..1) }\end{center}
or :
\begin{center}{\tt plotparam(sin(x)+i*cos(x),x=-4..1) }\end{center}
Output :
\begin{center}{\tt the arc (sin(-4)+i*cos(-4),sin(1)+i*cos(1)) of the unit circle}\end{center}
If in the graphic configuration {\tt t} goes from -4 to 1, input :
\begin{center}{\tt plotparam(sin(t)+i*cos(t),t,tstep=0.5)}\end{center}
Or :
\begin{center}{\tt plotparam(sin(t)+i*cos(t),t=-4..1,tstep=0.5)}\end{center}
Output :
\begin{center}{\tt A polygon approching the arc (sin(-4)+i*cos(-4),sin(1)+i*cos(1)) of the unit circle}\end{center}

\subsection{3D parametric surface : {\tt plotparam paramplot DrawParm}}\index{plotparam}\index{paramplot}\index{DrawParm}
\noindent{\tt plotparam} takes two main arguments,
a list of three 
expressions of two variables and the list of these variable names
where each variable name may be replaced by variable=interval
to specify the range of the parameters.
It accepts an optionnal argument to specify
the discretisation steps of the parameters $u$ and $v$ with 
{\tt ustep=...} and {\tt vstep=...}.\\
{\tt plotparam([f(u,v),g(u,v),h(u,v)],[u,v])} draws the surface defined by the 
first argument : $x=f(u,v),y=g(u,v),z=h(u,v)$, where $u$ and $v$
ranges default to the graphic configuration.\\
Input :
\begin{center}{\tt plotparam([v*cos(u),v*sin(u),v],[u,v])}\end{center}
Output :
\begin{center}{\tt The cone $x=v*\cos(u),y=v*\sin(u),z=v$}\end{center}
To specify the range of each parameters, replace each variable
by an equation variable=range, like this:
\begin{center}{\tt plotparam([v*cos(u),v*sin(u),v],[u=0..pi,v=0..3]) }\end{center}
Output :
\begin{center}{\tt A portion of the cone $x=v*\cos(u),y=v*\sin(u),z=v$}\end{center}
Input :
\begin{center}{\tt plotparam([v*cos(u),v*sin(u),v],[u=0..pi,v=0..3],ustep=0.5,vstep=0.5)}\end{center}
Output :
\begin{center}{\tt A portion of the cone $x=v*\cos(u),y=v*\sin(u),z=v$}\end{center}
 
\section{Curve defined in polar coordinates : {\tt plotpolar polarplot DrawPol courbe\_polaire}}\index{plotpolar|textbf}\index{polarplot|textbf}\index{DrawPol|textbf}\index{courbe\_polaire|textbf}
\noindent Let $E_t$ be an expression depending of the variable $t$.\\
{\tt plotpolar($E_t$,t)} draws the polar representation of the
curve defined by $\rho=E_t$ for $\theta=t$, that is
in cartesian coordinates the curve $(E_t \cos(t),E_t \sin(t))$.
The range of the parameter may be specified by replacing the second argument
by {\tt t=tmin..tmax}. The discretisation parameter may be specified
by an optional {\tt tstep=...} argument.\\
Input 
\begin{center}{\tt  plotpolar(t,t)}\end{center}
Output :
\begin{center}{\tt The spiral $\rho$=t is plotted}\end{center}
Input
\begin{center}{\tt  plotpolar(t,t,tstep=1)}\end{center}
or :
\begin{center}{\tt  plotpolar(t,t=0..10,tstep=1)}\end{center}
Output :
\begin{center}{\tt A polygon line approaching the spiral $\rho$=t is plotted}\end{center}

\section{Graph of a recurrent sequence : {\tt plotseq seqplot graphe\_suite}}\index{plotseq}\index{seqplot}\index{graphe\_suite}\label{sec:plotseq}
\noindent Let $f(x)$ be an expression depending of the variable $x$ 
(resp. $f(t)$ an expression depending of the variable $t$).\\
{\tt plotseq($f(x)$,a,n)} (resp. {\tt plotseq($f(t)$,t=a,n)}) draws the line 
$y=x$, the graph of $y=f(x)$ (resp $y=f(t)$) and the $n$ first terms of the
recurrent sequence defined by : $u_0=a,\ \ u_n=f(u_{n-1})$.
The $a$ value may be replaced by a list of 3 elements, $[a,x_-,x_+]$
where $x_-..x_+$ will be passed as $x$ range for the graph computation.\\ 
Input :
\begin{center}{\tt plotseq(sqrt(1+x),x=[3,0,5],5)}\end{center}
Output :
\begin{center}{\tt the graph of y=sqrt(1+x), of y=x and of the 5 first terms of the sequence u\_0=3 and u\_n=sqrt(1+u\_(n-1))}\end{center}

\section{Tangent field : {\tt plotfield fieldplot}}\index{plotfield}\index{fieldplot}
\begin{itemize}
\item Let $f(t,y)$ be an expression depending of two variables $t$ and $y$, 
then :
\begin{center}
{\tt plotfield(f(t,y),[t,y])}
\end{center} 
draws the tangent field of the 
differential equation $y'=f(t,y)$ where $y$ is a real variable and
where $t$ is the abscissa,
\item Let $V$ be 
a vector of two expressions depending of 2 variables $x,y$ but 
independant of the time $t$, then 
\begin{center}
{\tt plotfield(V,[x,y])}
\end{center}
draws the vector field $V$,
\item The range of values of $t,y$ or of $x,y$ can be specified with\\
{\tt t=tmin..tmax}, {\tt x=xmin..xmax}, {\tt y=ymin..ymax}\\
in place of the variable name.
\item The discretisation may be specified with optional
arguments {\tt xstep=...}, {\tt ystep=...}.
\end{itemize}
Input :
\begin{center}{\tt plotfield(4*sin(t*y),[t=0..2,y=-3..7]) }\end{center}
Output :
\begin{center}{\tt Segments with slope 4*sin(t*y), representing tangents, are plotting in different points}\end{center}
With two variables $x,y$, input :
\begin{center}
{\tt plotfield(5*[-y,x],[x=-1..1,y=-1..1]) }
\end{center}

\section{Plotting a solution of a differential equation : {\tt plotode odeplot}}\index{plotode}\index{odeplot}
\noindent Let $f(t,y)$ be an expression depending of two variables 
$t$ and $y$.
\begin{itemize}
\item {\tt plotode($f(t,y)$,[t,y],[t0,y0])} draws the solution of 
the differential equation $y'=f(t,y)$ crossing through 
the point {\tt (t0,y0)} (i.e. such that $y(t_0)=y_0$)
\item
By default, $t$ goes in both directions. The range of value of $t$
may be specified by the optional argument
{\tt t=tmin..tmax}.
\item
We can also represent, in the space or in the plane,
the solution of a differential equation 
$y'=f(t,y)$ where $y=(X,Y)$ is a vector of size 2.
Just replace  $y$ by the variable names $X,Y$
and the initial value $y_0$ by the two initial values of the
variables at time $t_0$.
\end{itemize}
Input :
\begin{center}{\tt plotode(sin(t*y),[t,y],[0,1]) }\end{center}
Output :
\begin{center}{The graph of the solution of y'=sin(t,y) crossing through the point (0,1)}\end{center}
Input~:
\begin{center}
{\tt S:=odeplot([h-0.3*h*p, 0.3*h*p-p], [t,h,p],[0,0.3,0.7])}
\end{center}
Output, the graph in the space of the solution of :
\[ [h,p]'=[h-0.3 h*p, 0.3 h*p-p] \quad [h,p](0)=[0.3,0.7] \]
To have a 2-d graph (in the plane), use the option 
{\tt plane}
\begin{center}
{\tt S:=odeplot([h-0.3*h*p, 0.3*h*p-p], [t,h,p],[0,0.3,0.7],plane)}
\end{center}
To compute the values of the solution, see
the section \ref{sec:odesolve}.

\section{Interactive plotting of solutions of a differential equation : {\tt interactive\_plotode interactive\_odeplot}}\index{interactive\_plotode}\index{interactive\_odeplot}
\noindent Let $f(t,y)$ be an expression depending of two 
varaiables $t$ and $y$.\\
{\tt interactive\_plotode(f(t,y),[t,y])} draws the tangent field
of the differential equation $y'=f(t,y)$ in a new window. 
In this window, one can click on a point to get the 
plot of the solution of $y'=f(t,y)$ crossing through this point.\\
You can further click to display 
several solutions. To stop  press
the {\tt Esc} key.\\
Input :
\begin{center}{\tt interactive\_plotode(sin(t*y),[t,y]) }\end{center}
Output :
\begin{center}{\tt The tangent field is plotted with the
    solutions of y'=sin(t,y) crossing through the points defined by
     mouse clicks}\end{center}

\section{Animated graphs (2D, 3D or "4D")}
{\tt Xcas} can display animated 2D, 3D or "4D" graphs. 
This is done first by computing
a sequence of graphic objects, then after completion,
by displaying the sequence in a loop.
\begin{itemize} 
\item To stop or start again the animation, click on the button 
$\blacktriangleright \mid$ (at the left of {\tt Menu}).
\item
The display time of each graphic objet is specified in {\tt animate} of the
graph configuration ({\tt cfg} button). Put a small time, 
to have a fast animation.
\item
If {\tt animate} is {\tt 0}, the animation is frozen,
you can move in the sequence of objects one by one by clicking
on the mouse in the graphic scene.
\end{itemize}

\subsection{Animation of a 2D graph~: {\tt animate}}\index{animate}
\noindent{\tt animate} can create a 2-d animation with graphs of functions
depending of a parameter. The parameter is specified as the 
third argument of 
{\tt animate}, the number of pictures as fourth argument with
{\tt frames=}\index{frames@{\sl frames}|textbf}number, 
the remaining arguments are the same as those of the {\tt plot} command, 
see section \ref{sec:plot2d}, p. \pageref{sec:plot2d}.\\
Input :
\begin{center}
{\tt animate(sin(a*x),x=-pi..pi,a=-2..2,frames=10,color=red)}
\end{center}
Output :
\begin{center}{\tt a sequence of graphic representations of y=sin($a$x) for 
11 values of $a$ between -2 and 2}\end{center}

\subsection{Animation of a 3D graph~: {\tt animate3d}}\index{animate3d}
\noindent{\tt animate3d} can create a 3-d animation with 
function graphs depending of a parameter. The parameter is specified as
the third argument of {\tt animate3d}, the number of pictures
as fourth argument with 
{\tt frames=}\index{frames@{\sl frames}}number, the remaining arguments
are the same as those of the {\tt plotfunc} command, see
section \ref{sec:plotfunc3}, p. \pageref{sec:plotfunc3}.\\
Input :
\begin{center}
{\tt animate3d(x\verb|^|2+a*y\verb|^|2,[x=-2..2,y=-2..2],a=-2..2, frames=10,display=red+filled)}
\end{center}
Output :
\begin{center}{\tt a sequence of graphic representations
 of z=x\verb|^|2+$a$*y\verb|^|2 for 11 values of $a$ between -2 and 2}
\end{center}

\subsection{Animation of a sequence of graphic objects~: {\tt animation}}\index{animation}
\noindent{\tt animation} animates the representation of a
sequence of graphic objects
with a given display time. The sequence of objects depends most of
the time of a parameter and is defined using the {\tt seq} command
but it is not mandatory.\\
{\tt animation} takes as argument the sequence of graphic objects.\\
To define a sequence of graphic objects with {\tt seq},
enter the definition of the graphic object (depending on
the parameter), the parameter name, it's minimum value, it's
 maximum value maximum and optionnaly a step value.\\
Input :
\begin{center}{\tt animation(seq(plotfunc(cos(a*x),x),a,0,10))}\end{center}
Output :
\begin{center}{\tt The sequence of the curves defined by $y=\cos(ax)$, for $a=0,1,2..10$}\end{center}
Input :
\begin{center}
{\tt animation(seq(plotfunc(cos(a*x),x),a,0,10,0.5))}\\
or\\
{\tt animation(seq(plotfunc(cos(a*x),x),a=0..10,0.5))}
\end{center}
Output :
\begin{center}{\tt The sequence of the curves defined by $y=\cos(ax)$, for $a=0,0.5,1,1.5..10$ }\end{center}
Input :
\begin{center}{\tt animation(seq(plotfunc([cos(a*x),sin(a*x)],x=0..2*pi/a), a,1,10))}\end{center}
Output :
\begin{center}{\tt The sequence of two curves defined by $y=\cos(ax)$ and $y=\sin(ax)$, for $a=1..10$ and for $x=0..2\pi/a$ }\end{center}
Input :
\begin{center}{\tt animation(seq(plotparam([cos(a*t),sin(a*t)], t=0..2*pi),a,1,10))}\end{center}
Output :
\begin{center}{\tt The sequence of the parametric curves defined by  $x=\cos(at)$ and $y=\sin(at)$, for $a=1..10$ and for $t=0..2\pi$ }\end{center}
Input :
\begin{center}{\tt animation(seq(plotparam([sin(t),sin(a*t)], t,0,2*pi,tstep=0.01),a,1,10))}\end{center}
Output :
\begin{center}{\tt The sequence of the parametric curves defined by $x=\sin(t),y=\sin(at)$, for $a=0..10$ and $t=0..2\pi$}\end{center}
Input :
\begin{center}{\tt animation(seq(plotpolar(1-a*0.01*t\verb|^|2, t,0,5*pi,tstep=0.01),a,1,10))}\end{center}
Output :
\begin{center}{\tt The sequence of the polar curves defined by $\rho=1-a*0.01*t^2$, for $a=0..10$ and $t=0..5\pi$}\end{center}
Input :
\begin{center}{\tt plotfield(sin(x*y),[x,y]); animation(seq(plotode(sin(x*y),[x,y],[0,a]),a,-4,4,0.5))}\end{center}
Output :
\begin{center}{\tt The tangent field of y'=sin(xy) and the sequence of the integral curves crossing through the point $(0,a)$ for $a$=-4,-3.5...3.5,4}\end{center}
Input :
\begin{center}{\tt animation(seq(display(square(0,1+i*a),filled),a,-5,5))}\end{center}
Output :
\begin{center}{\tt The sequence of the squares defined by the points 0 and 1+i*$a$ for $a=-5..5$}\end{center}
Input :
\begin{center}{\tt animation(seq(droite([0,0,0],[1,1,a]),a,-5,5))}\end{center}
Output :
\begin{center}{\tt The sequence of the lines defined by the points [0,0,0] and [1,1,$a$] for $a=-5..5$}\end{center}
Input :
\begin{center}{\tt animation(seq(plotfunc(x\verb|^|2-y\verb|^|a,[x,y]),a=1..3))}\end{center}
Output :
\begin{center}{\tt The sequence of the "3D" surface defined by $x^2-y^a$, for $a=1..3$ with rainbow colors}\end{center}
Input :
\begin{center}{\tt animation(seq(plotfunc((x+i*y)\verb|^|a,[x,y], display=filled),a=1..10)}\end{center}
Output :
\begin{center}{\tt The sequence of the "4D" surfaces defined by $(x+i*y)^a$, for $a=0..10$ with rainbow colors}\end{center}

{\bf Remark}
We may also define the sequence with a program, 
for example if we want to draw the
segments of length $1,\sqrt 2...\sqrt 20$ constructed with a 
right triangle of side 1 and the previous segment
(note that there is a {\tt c:=evalf(..)} statement
to force approx. evaluation otherwise the computing time 
would be too long) :
\begin{verbatim}
seg(n):={
 local a,b,c,j,aa,bb,L;
 a:=1;
 b:=1;
 L:=[point(1)];
 for(j:=1;j<=n;j++){
  L:=append(L,point(a+i*b));
  c:=evalf(sqrt(a^2+b^2));
  aa:=a;
  bb:=b;
  a:=aa-bb/c;
  b:=bb+aa/c;
 }
 L;
}
\end{verbatim}
Then input : 
\begin{center}{\tt animation(seg(20))}\end{center}
We see, each point, one to one with a display time that
depends of the {\tt animate} value in {\tt cfg}.\\
Or :
\begin{center}{\tt L:=seg(20); s:=segment(0,L[k])\$(k=0..20)}\end{center}
We see 21 segments. \\
Then, input :
\begin{center}{\tt animation(s)}\end{center}
We see, each segment, one to one with a display time that
depends of the {\tt animate} value in {\tt cfg}.



\end{document}